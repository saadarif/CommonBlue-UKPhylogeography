\documentclass[12pt,]{article}
\usepackage{lmodern}
\usepackage{amssymb,amsmath}
\usepackage{ifxetex,ifluatex}
\usepackage{fixltx2e} % provides \textsubscript
\ifnum 0\ifxetex 1\fi\ifluatex 1\fi=0 % if pdftex
  \usepackage[T1]{fontenc}
  \usepackage[utf8]{inputenc}
\else % if luatex or xelatex
  \ifxetex
    \usepackage{mathspec}
  \else
    \usepackage{fontspec}
  \fi
  \defaultfontfeatures{Ligatures=TeX,Scale=MatchLowercase}
\fi
% use upquote if available, for straight quotes in verbatim environments
\IfFileExists{upquote.sty}{\usepackage{upquote}}{}
% use microtype if available
\IfFileExists{microtype.sty}{%
\usepackage{microtype}
\UseMicrotypeSet[protrusion]{basicmath} % disable protrusion for tt fonts
}{}
\usepackage[margin=1in]{geometry}
\usepackage{hyperref}
\hypersetup{unicode=true,
            pdfborder={0 0 0},
            breaklinks=true}
\urlstyle{same}  % don't use monospace font for urls
\usepackage{graphicx,grffile}
\makeatletter
\def\maxwidth{\ifdim\Gin@nat@width>\linewidth\linewidth\else\Gin@nat@width\fi}
\def\maxheight{\ifdim\Gin@nat@height>\textheight\textheight\else\Gin@nat@height\fi}
\makeatother
% Scale images if necessary, so that they will not overflow the page
% margins by default, and it is still possible to overwrite the defaults
% using explicit options in \includegraphics[width, height, ...]{}
\setkeys{Gin}{width=\maxwidth,height=\maxheight,keepaspectratio}
\IfFileExists{parskip.sty}{%
\usepackage{parskip}
}{% else
\setlength{\parindent}{0pt}
\setlength{\parskip}{6pt plus 2pt minus 1pt}
}
\setlength{\emergencystretch}{3em}  % prevent overfull lines
\providecommand{\tightlist}{%
  \setlength{\itemsep}{0pt}\setlength{\parskip}{0pt}}
\setcounter{secnumdepth}{0}
% Redefines (sub)paragraphs to behave more like sections
\ifx\paragraph\undefined\else
\let\oldparagraph\paragraph
\renewcommand{\paragraph}[1]{\oldparagraph{#1}\mbox{}}
\fi
\ifx\subparagraph\undefined\else
\let\oldsubparagraph\subparagraph
\renewcommand{\subparagraph}[1]{\oldsubparagraph{#1}\mbox{}}
\fi

%%% Use protect on footnotes to avoid problems with footnotes in titles
\let\rmarkdownfootnote\footnote%
\def\footnote{\protect\rmarkdownfootnote}

%%% Change title format to be more compact
\usepackage{titling}

% Create subtitle command for use in maketitle
\providecommand{\subtitle}[1]{
  \posttitle{
    \begin{center}\large#1\end{center}
    }
}

\setlength{\droptitle}{-2em}

  \title{}
    \pretitle{\vspace{\droptitle}}
  \posttitle{}
    \author{}
    \preauthor{}\postauthor{}
    \date{}
    \predate{}\postdate{}
  
\usepackage{fancyhdr}
\usepackage{lastpage}
\usepackage{lineno}
\linenumbers

\begin{document}

\fancypagestyle{plain}{%
  \renewcommand{\headrulewidth}{0pt}%
  \fancyhf{}%
  \fancyfoot[R]{\footnotesize Page \thepage\, of\, \pageref*{LastPage}}
  \setlength\footskip{0pt}
} \pagestyle{plain}

\begin{center}\includegraphics{FigCompile_files/figure-latex/Fig1-1} \end{center}

\textbf{Figure 1} Geographical locations of Great British
\emph{Polyommatus icarus} individuals sampled for this study. An
additional 6 individuals collected from southern France are not shown
(see Table S2 for full details). Size of circles is proportional to the
number of samples acquired at each locality.

\includegraphics{FigCompile_files/figure-latex/Fig2-1}

\textbf{Figure 2 (A)} Bayesian ultrametric tree inferred from 187 CO1
\emph{Polyommatus icarus} sequences. Numbers above nodes represent
Bayesian posterior support (only where probabilities were \textgreater{}
0.75). The bars on each node represent the 95\% Bayesian credible
interval for estimates of node age. The scale is in millions of year.
Clades with \textgreater{} 0.75 posterior support have been collapsed
(see Figure S3 for uncollpased tree) and coloured based on
correspondence to the \emph{CO1} lineage classification from Dincă
\emph{et al.} (2011). \textbf{(B)} Bubble plot of the geographical
distribution of the 51 Great British samples in the dataset. Colours are
based on their correspondence to the lineage classification from Dincă
\emph{et al.} (2011). Size of the bubble represents the total number of
CO1 lineages at a particular site.

\pagebreak

\includegraphics{FigCompile_files/figure-latex/Fig3-1}

\textbf{Figure 3} Principal component analysis based on 2824 unlinked
SNPs and 148 individuals. PC1 explains 6.8\% of the variation in genomic
SNPs and PC2 explains 1.7\%. Ninety-\% Confidence Interval (CI) ellipses
for PC1 and 2 for each locality are also shown. Each locality is
represented by a distinct colour and inset shows a labelled map of the
sampled localities for quick reference.

\begin{center}\includegraphics{FigCompile_files/figure-latex/Fig4-1} \end{center}

\textbf{Figure 4 (A)} Proportion of individuals classified as infected
based on a threshold of log2 (percentage of classified reads mapping to
\emph{Wolbachia}) \textgreater{} 0. Only populations with at least one
infected individual are shown. Error bars are the standard errors of the
estimated proportions. Inset shows the geographical range of populations
with infected individuals. \textbf{(B)} Distributions of the percentage
of classified reads mapping to \emph{Wolbachia} for locations with
individuals classified as infected. RHD with a single infected male is
not shown. Larger circles represent the average for each sex within
locations.

\pagebreak

\begin{center}\includegraphics{FigCompile_files/figure-latex/Fig5-1} \end{center}

\textbf{Figure 5} UPGMA clustering of mitochondrial CO1 fragments from
30 \emph{Polyommatus icarus} individuals based on bitwise distance
(left). UPGMA clustering of 68 concatenated SNPs from ddRADseq
\emph{Wolbachia} genotypes derived from reads mapping to Wolbachia from
42 infected individuals (right). Numbers on nodes represent bootstrap
branch support values based on 1000 bootstrap replicates, values
\textless{} 70\% are not shown. Large circles represent individuals from
mainland populations (DGC, MLG, OBN, RHD) and smaller circles represent
individuals from the Outer Hebrides (BER, TUL). Lines between dendograms
connect the CO1 haplotype and \emph{Wolbachia} straind for the same
individual. Scale bar reflects the proportion of loci that are
different.

\pagebreak

\begin{center}\includegraphics{FigCompile_files/figure-latex/Fig6-1} \end{center}

\textbf{Figure 6} Feminization of \emph{w}Ica1 infected females as
suggested by discordance between phenotypic sex, based on external
genitalia and dimorphic wing patterning, and genetic sex based on
female-specific markers. Discordance is exemplified by data on a single
marker here (11011\_27), see Table S7 for more examples. All
\emph{w}Ica1 infected morphological females (all females in the Outer
Hebrides and one from the mainland(MLGf010)) are homozygous for
female-specific markers, consistent with feminization.


\end{document}
